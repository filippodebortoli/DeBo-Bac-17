\section{Fixed point theory for partially ordered sets}

\begin{dfn}
  A \emph{partially ordered set} --- \emph{poset} --- is a pair
  \((C, \lhd)\), where \(C\) is a set and \({}\lhd{}\) is a binary relation
  on it which is reflexive, antisymmetric and transitive.
\end{dfn}

In a poset \((C, {}\lhd)\), an element \(c \in C\) is a \emph{minimal}
element of \(C\) iff there isn't any \(y \in C\) different from \(x\) such
that \(y \lhd x\), and it is a \emph{least} one iff each \(y \in C\) satisfies
\(x \lhd y\).
Similarly, we can define the notions of \emph{maximal} and \emph{greatest}
elements of a poset.

% Saltati un po' di pezzi.

\begin{dfn}
  An \(\omega{}\)-\emph{complete partial order} --- \(\omega{}\)-\emph{cpo} ---
  is a poset with a least element in which every ascending \(\omega{}\)-chain
  has a least upper bound in it.
\end{dfn}

\begin{dfn}
  Let \((C, {}<)\) and \((C^\prime, {}\lhd)\) be posets.
  A function \(f \colon C \to C^\prime\) is \emph{monotonic} iff
  \[
    \forall x,y \in C.\, x \lhd y \implies f(x) \lhd f(y).
  \]
\end{dfn}

\begin{dfn}
  Let \(C\) and \(C^\prime\) be \(\omega{}\)-cpos.
  A function \(f \colon C \to C^\prime\) is \(\omega{}\)-\emph{continuous}
  iff it is monotonic and, for each ascending \(\omega{}\)-chain
  \(\lbrace x_i \rbrace_{i=0}^\infty\) in \(C\),
  \[
  f\left(\bigcup_{i=0}^\infty x_i\right) = \bigcup_{i=0}^\infty f(x_i).
  \]
\end{dfn}

\begin{dfn}
  Given an endofunction \(f\) on the poset \((C,<)\), \(x \in C\) is a \emph{prefixed} point of \(f\) if \(f(x) < x\), and it is the \emph{least prefixed point} if \(x < y\) for each \(y\) prefixed point of \(f\).
  Similarly, a \emph{fixed} point of \(f\) is an element \(x \in C\) such that \(f(x) = x\); it is the \emph{least fixed point} if \(x < y\) for each \(y\) fixed point of \(f\).
\end{dfn}

\begin{lem}
  Let \(f\) be a monotonic endofunction on a poset \((C,\le)\).
  Then, the least prefixed point of \(f\) is also its least fixed point.
\end{lem}
\begin{proof}
  Being \(f\) monotonic, assuming that \(x \in C\) is its least prefixed point, we can derive that
  \begin{equation*}
    f(x) \le x \implies f(f(x)) \le f(x),
  \end{equation*}
  so \(f(x)\) is also a prefixed point of \(f\).
  Then, it follows that \(x \le f(x)\) and consequently \(x = f(x)\), i.e \(x\) is a fixed point of \(f\).
  Moreover, it is also the least fixed point, because every fixed point is also a prefixed point of \(f\), and \(x\) is assumed to be the least prefixed point of \(f\).  
\end{proof}

\begin{thm}[Knaster--Tarski]
  If \(f\) is a monotonic function, then it has a least fixed point.
\end{thm}
\begin{proof}[Da rivedere]
  Let \(\mathcal{F}\) be the set of prefixed points of \(f\), and let \(x = \bigsqcap\mathcal{F}\). Then \(x \sqsubseteq \bigsqcap\mathcal{F}\), and, thanks to monotonicity,
  \begin{equation*}
    \left(\forall y \in \mathcal{F}.\, x \sqsubseteq y\right) \implies \forall y \in \mathcal{F}.\,f(x) \sqsubseteq f(y)
  \end{equation*}
  Moreover, being each \(y \in \mathcal{F}\) itself a prefixed point, it follows that \(f(x) \sqsubseteq y\), thus \(f(x) \sqsubseteq \cap\mathcal{F}\).
  Finally, we conclude that \(f(x) \sqsubseteq x\), i.e. \(x\) is the least prefixed point of \(f\).
\end{proof}

\begin{thm}[Kleene]
  Let \(f\) be a Scott--continuous endofunction on a complete lattice \((C,\sqsubseteq)\) --- i.e. \((C,\sqsubseteq)\) is a poset in which every subset of \(C\) has a least upper bound in \(C\). Then, \(\lbrace f^n(\bot) \rbrace_{n \in \mathbb{N}}\) is an ascending \(\omega\)--chain and
  \begin{equation}
    \fix(f) = \bigsqcup_{n \in \mathbb{N}} f^n(\bot)
  \end{equation}
  is the least fixed point of \(f\).
\end{thm}
\begin{proof}
  We begin by proving that \(\lbrace f^n(\bot) \rbrace_{n \in \mathbb{N}}\) is indeed an ascending \(\omega\)--chain, showing by induction on \(n \in \mathbb{N}\) that \(f^n(\bot) \sqsubseteq f^{n+1}(\bot)\).
  \begin{description}
    \item[Induction basis] for \(n = 0\), \(\bot = f^0(\bot) \sqsubseteq f^1(\bot) = f(\bot)\) vacuously holds, since \(\bot\) is the least element of \(C\).
    \item[Induction step] assuming that \(f^n(\bot) \sqsubseteq f^{n+1}(\bot)\) holds for \(n \in \mathbb{N}\), we obtain that
    \begin{equation*}
      \begin{split}
        f^n(\bot) \sqsubseteq f^{n+1}(\bot) &\implies
        f(f^n(\bot)) \sqsubseteq f(f^{n+1}(\bot)) \\
        &\implies
        f^{n+1}(\bot) \sqsubseteq f^{(n+1)+1}(\bot),
      \end{split}
    \end{equation*}
    since \(f\) is monotonic.
  \end{description}
  
  Now, (proseguiamo un'altra volta!)
\end{proof}