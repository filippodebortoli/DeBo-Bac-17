\section{\(\mathbf{O}\)--categories and projection pairs}

In a general setting, verifying whether a given category is \(\omega\)--cocomplete --- and, in a similar fashion, checking if a functor is \(\omega\)--cocontinuous --- is not an easy task. %chktex 21
Nevertheless, there is a class of categories, first introduced by Wand in~\cite{Wand1979}, which structure allows for easily verifiable criteria to arise.
These so--called \(\mathbf{O}\)--categories include almost every category that occurs in semantics.%chktex 8
%
% This section goes along with the work presented in~\cite{Smyth1982}.

\begin{dfn}
  An \emph{\(\mathbf{O}\)--category} is a category which satisfies the following conditions:
  \begin{enumerate}
    \item every hom--set is a partial order~\footnote{The partial order on a hom--set in an \(\mathbf{O}\)--category will hereafter be denoted by \(\sqsubseteq{}\).} in which every ascending %chktex 8
    \(\omega\)--chain has a least upper bound; %chktex 21
    \item composition of morphisms is an \(\omega\)--continuous operation with respect to that partial order.%chktex 21
  \end{enumerate}
\end{dfn}

In an \(\mathbf{O}\)--category, it is possible to consider \emph{projection pairs}. These turn out to have some useful applications, which we are going to show in later sections.

\begin{dfn}
  Let \(\mathcal{C}\) be an \(\mathbf{O}\)--category. Then, every tuple \((f,f^\star)\) satisfying
  \begin{align}
    f &\colon A \to B & f^\star \circ f = 1_A \quad  \\
    f^\star &\colon B \to A & f \circ f^\star \sqsubseteq 1_B \quad
  \end{align}
  with \(A,B \in \obj{\mathcal{C}}\) is called a \emph{projection pair}, where \(f\) is an \emph{embedding} and \(f^\star\) a \emph{projection}.%chktex 21
\end{dfn}

An interesting feature of projection pairs is that they're uniquely determined by one of their components: given an embedding, we can thereby take \emph{the} associated projection.
The following lemma justifies what has just been asserted.

\begin{lem}
  Let \((f,f^\star)\) and \((g,g^\star)\) be projection pairs from \(A\) to \(B\).
  Then,
  \begin{equation*}
    f \sqsubseteq g \iff g^\star \sqsubseteq f^\star \qquad \text{and} \qquad
    f = g \iff f^\star = g^\star.
  \end{equation*}
\end{lem}
\begin{proof}
  Assume that \(f \sqsubseteq h\). Since composition is \(\omega{}\)--continuous --- hence monotonic --- we find
  \begin{equation*}
    g^\star = 1_A \circ g^\star = (f^\star \circ f) \circ g^\star
    \sqsubseteq (f^\star \circ g) \circ g^\star \sqsubseteq f^\star \circ 1_B = f^\star.
  \end{equation*}
  Similarly, we prove the opposite implication. Finally,
  \begin{equation*}
    f = g \iff f \sqsubseteq g \land g \sqsubseteq f
    \iff g^\star \sqsubseteq f^\star \land f^\star \sqsubseteq g^\star \iff f^\star = g^\star.\qedhere
  \end{equation*}
\end{proof}

For each embedding \(f\), we will denote by \(f^\star\) its associated projection. %chktex 21
In addition, for an \(\mathbf{O}\)--category \(\mathcal{C}\), we denote by \(\mathcal{C}^\textup{E}\) its subcategory consisting of all its objects, but with as morphisms only the embeddings.
In fact, the composition of embeddings is still an embedding: given embeddings \(f\), \(g\) the arrow \(g \circ f\) is an embedding with corresponding projection \({(g \circ f)}^\star = f^\star \circ g^\star\).%chktex 21
In a similar fashion, it is possible to define \(\mathcal{C}^\textup{P}\), the subcategory of \(\mathcal{C}\) which morphisms are projections.

\subsection{The initiality theorem}

A first, important result, concerning colimits in \(\mathbf{O}\)--categories, is stated in the initiality theorem, that provides a simple way to check whether a cocone is a colimit in a subclass of \(\mathbf{O}\)--categories.

\begin{lem}\label{lem:3-1}
  Let \(\mathcal{C}\) be an \(\mathbf{O}\)--category, \(\Delta{}\) an \(\omega{}\)--chain in \(\mathcal{C}^\textup{E}\) and \((D,\alpha)\), \((E,\beta)\) two \(\Delta\)--cocones.%chktex 21
  Then, the following facts hold:
  \begin{enumerate}
    \item \(\lbrace \alpha_i \circ \beta_i^\star \rbrace_{i \in \mathbb{N}}\) is an ascending \(\omega\)--chain in \(\mathcal{C}(E,D)\);%chktex 21
    \item if \(f = \bigsqcup_{i \in \mathbb{N}}(\alpha_i \circ \beta_i^\star)\), then \(f\) is a morphisms of cocones:
    \begin{equation*}
      \forall n \in \mathbb{N}.\: f \circ \beta_n = \alpha_n.
    \end{equation*}
  \end{enumerate}
\end{lem}
\begin{proof}
  Let \(n \in \mathbb{N}\) such that \(n \ge 0\). Then,
  \begin{equation*}
    \begin{split}
      \alpha_n \circ \beta_n^\star &=
      (\alpha_{n+1} \circ f_n) \circ {(\beta_{n+1} \circ f_n)}^\star \\
      &= \alpha_{n+1} \circ (f_n \circ f_n^\star) \circ \beta_{n+1}^\star
      \sqsubseteq \alpha_{n+1} \circ \beta_{n+1}.
    \end{split}
  \end{equation*}

  Passing on to the second fact, we should first note that \(f\) is well--defined, since \(\mathcal{C}\) is an \(\mathbf{O}\)--category. %chktex 8
  Furthermore, the deletion of an initial segment of an \(\omega\)--chain leaves its least upper bound unaffected. %chktex 21
  So,
  \begin{equation*}
    \begin{split}
      f \circ \beta_n &=
      \bigsqcup_{i \in \mathbb{N}}(\alpha_i \circ \beta_i^\star) \circ \beta_n
      = \bigsqcup_{i \ge n}(\alpha_i \circ \beta_i^\star) \circ \beta_n \\
      &= \bigsqcup_{i \ge n}(\alpha_i \circ \beta_i^\star \circ \beta_n)
      = \bigsqcup_{i \ge n}(\alpha_i \circ \beta_i^\star \circ \beta_i \circ f_i \circ f_{i+1} \circ \dotsc \circ f_n) \\
      &= \bigsqcup_{i \ge n}(\alpha_i \circ f_i \circ f_{i+1} \circ \dotsc \circ f_n)
      = \bigsqcup_{i \ge n}(\alpha_n) = \alpha_n.
    \end{split}
  \end{equation*}\qedhere
\end{proof}

A corner--case of the Lemma~\ref{lem:3-1} occurs when the two cocones coincide, i.e. \((E,\beta) = (D,\alpha)\). %chktex 8
Of course, \(f = \bigsqcup_{i \in \mathbb{N}}(\alpha_i \circ \alpha_i^\star) \sqsubseteq 1_D\).
Is it possible to make additional claims, if equality is reached?

\begin{thm}\label{thm:3-1}
  Let \(\mathcal{C}\) be an \(\mathbf{O}\)--category, \(\Delta = \lbrace (D_i,f_i) \rbrace_{i \in \mathbb{N}}\) an \(\omega{}\)--chain in \(\mathcal{C}^\textup{E}\) and \((D,\alpha)\) a cocone for \(\Delta{}\).
  Suppose that \(\bigsqcup_{i \in \mathbb{N}}(\alpha_i \circ \alpha_i^\star) = 1_D\).
  Then, \((D,\alpha)\) is a colimit for \(\Delta{}\).
  Moreover, if \((E,\beta)\) is a cocone for \(\Delta{}\), then
  \begin{equation*}
    f = \bigsqcup_{i \in \mathbb{N}}(\beta_i \circ \alpha_i^\star)
  \end{equation*}
  is the mediating morphism from \(D\) to \(E\) --- and its associated projection is \(f^\star = \bigsqcup_{i \in \mathbb{N}}(\alpha_i \circ \beta_i^\star)\).
\end{thm}
\begin{proof}
  First, we are going to show that \(f\) is an embedding from \(D\) to \(E\).
  \begin{align*}
    f^\star \circ f &\overset{\bullet}{=}
    \bigsqcup_{i \in \mathbb{N}}(\alpha_i \circ \beta_i^\star \circ \beta_i \circ \alpha_i^\star) =
    \bigsqcup_{i \in \mathbb{N}}(\alpha_i \circ \alpha_i^\star) \overset{\dagger}{=} 1_D; \\
    f \circ f^\star &\overset{\bullet}{=}
    \bigsqcup_{i \in \mathbb{N}}(\beta_i \circ \alpha_i^\star \circ \alpha_i \circ \beta_i^\star) =
    \bigsqcup_{i \in \mathbb{N}}(\beta_i \circ \beta_i^\star)
    \sqsubseteq 1_E.
  \end{align*}
  The equalities marked with \(\bullet\) are justified from the \(\omega\)--continuity of the composition, while \(\dagger\) follows from the hypothesis. %chktex 21

  As already seen in Lemma~\ref{lem:3-1}, \(f\) is a morphism of cocones.
  Now, what is left is proving that \(f\) is the unique morphism from \((D,\alpha)\) to \((E,\beta)\) such that, for all \(n \in \mathbb{N}\), \(f \circ \alpha_n = \beta_n\).
  Let \(\eta\) be another such morphism. Then, %chktex 21
  \begin{equation*}
    \begin{split}
      \eta &= \eta \circ 1_D
      = \eta \circ \bigsqcup_{i \in \mathbb{N}}(\alpha_i \circ \alpha_i^\star) = \bigsqcup_{i \in \mathbb{N}}(\eta \circ \alpha_i \circ \alpha_i^\star) \\
      &= \bigsqcup_{i \in \mathbb{N}}(\beta_i \circ \alpha_i^\star)
      = f.
    \end{split}
  \end{equation*}
\end{proof}

It would be desirable to have a converse result, with respect to Theorem~\ref{thm:3-1}, namely, showing that \((D,\alpha)\) is a colimit if and only if the identity morphism \(1_D\) equals \(\bigsqcup_{i \in \mathbb{N}}(\alpha_i \circ \alpha_i^\star)\).

\begin{dfn}\label{dfn:localized}
  An \(\mathbf{O}\)--category \(\mathcal{C}\) is called \emph{localized} if and only if for any \(\omega{}\)--chain \(\Delta{}\) in \(\mathcal{C}^\textup{E}\) and any \(\Delta{}\)--colimit \((D,\alpha)\) there exist an object \(E \in \obj{\mathcal{C}}\) and a projection pair \((i,g) \colon E \to D\) such that
  \begin{equation*}
    \bigsqcup_{n \in \mathbb{N}}(\alpha_n \circ \alpha_n^\star) = i \circ g.
  \end{equation*}
\end{dfn}

Being \((i,g)\) a projection pair, \(i \circ g \sqsubseteq 1_D\) vacuously holds. If the limit--case \(i \circ g = 1_D\) is reached, then \(E\) and \(D\) are isomorphic and \(g = i^{-1}\). %chktex 8
Conversely, if \((i,g)\) has an inverse \(\eta{}\), i.e. \((i,g) \circ \eta = (1_D,1_D)\), then
\begin{equation*}
    1_D = 1_D \circ 1_D = i \circ \eta^\textup{L} \circ \eta^\star \circ g
    \sqsubseteq i \circ g;
\end{equation*}
this, jointly with \(i \circ g \sqsubseteq 1_D\), leads to \(i \circ g = 1_D\).

\begin{thm}[Initiality theorem]\label{thm:3-2}
  Let \(\mathcal{C}\) be a localized \(\mathbf{O}\)--category, \(\Delta{}\) an \(\omega{}\)--chain in \(\mathcal{C}^\textup{E}\) and \((D,\alpha)\) a cocone for \(\Delta{}\). The following are equivalent:
  \begin{enumerate}
    \item \((D,\alpha)\) is a \(\Delta{}\)--colimit;
    \item \(\bigsqcup_{i \in \mathbb{N}}(\alpha_i \circ \alpha_i^\star) = 1_D\).
  \end{enumerate}
\end{thm}
\begin{proof}
  Again, to be seen later.
\end{proof}

To test whether an \(\mathbf{O}\)--category is localized, it would be desirable to have a simpler criterion. This can be obtained, thanks to idempotents and splits. Let \(\mathcal{C}\) be a category and \(f \colon D \to D\) an arrow, with \(D \in \obj{\mathcal{C}}\). Then, \(f\) is called an \emph{idempotent} when \(f \circ f = f\); it is called \emph{split} when there exists another object \(E\) and morphisms \(g \colon D \to E\), \(h \colon E \to D\) such that \(f = h \circ g\) and \(g \circ h = 1_E\).
\begin{center}
  \begin{tikzcd}
    & E \arrow[r, "1_E"] \arrow[d, "h"] & E \\
    D \arrow[r, "f"] \arrow[ur, "g"] & D \arrow[ur, "g"] &
  \end{tikzcd}
\end{center}

Note that any split is idempotent.\footnote{Indeed, \(f \circ f = h \circ g \circ h \circ g = h \circ 1_E \circ g = f\).}

\begin{prp}
  If every idempotent in an \(\mathbf{O}\)--category \(\mathcal{C}\) is split, then \(\mathcal{C}\) is localized.
\end{prp}
\begin{proof}
  Let \((D,\alpha)\) be a cocone for an \(\omega{}\)--chain \(\Delta{}\) in \(\mathcal{C}^\textup{E}\).
  Then, the morphism \(f := \bigsqcup_{i \in \mathbb{N}}(\alpha_i \circ \alpha_i^\star)\) is idempotent:
  \begin{equation*}
    \begin{split}
      f \circ f &= (\bigsqcup_{i \in \mathbb{N}}(\alpha_i \circ \alpha_i^\star)) \circ (\bigsqcup_{i \in \mathbb{N}}(\alpha_i \circ \alpha_i^\star)) \\
      &= \bigsqcup_{i \in \mathbb{N}}(\alpha_i \circ \alpha_i^\star \circ \alpha_i \circ \alpha_i^\star)
      = \bigsqcup_{i \in \mathbb{N}}(\alpha_i \circ \alpha_i^\star) = f.
    \end{split}
  \end{equation*}
  Thanks to the assumption that every idempotent is split in \(\mathcal{C}\), there exist an object \(E\) and two morphisms \(g \colon D \to E\), \(h \colon E \to D\) such that \(g \circ h = 1_E\) and \(h \circ g = f \sqsubseteq 1_D\).
  Thus, \((g,h)\) is a projection pair from \(E\) to \(D\) with \(f = h \circ g\).
  This makes us conclude that \(\mathcal{C}\) is localized.
\end{proof}

\subsection{The continuity theorem}

Let \(\mathcal{C}\) and \(\mathcal{L}\) be \(\mathbf{O}\)--categories. Suppose that \(F \colon \mathcal{C} \to \mathcal{L}\) is a functor.
Which conditions should we impose, to ensure that a functor \(F^\textup{E} \colon \mathcal{C}^\textup{E} \to \mathcal{L}^\textup{E}\) can be induced?

One would start by define \(F^\textup{E}\) as follows:
\begin{align*}
  F^\textup{E}(A) &= FA, A \in \obj{\mathcal{C}^\textup{E}} \\
  F^\textup{E}(f) &= Ff, f\:\text{a morphism of}\:\mathcal{C}^\textup{E}.
\end{align*}
Is \(F(f)\) a morphism in \(\mathcal{L}^\textup{E}\)?
That is, assuming \((f,g) \colon D \to E\) is a projection pair in \(\mathcal{C}\), is \((Ff,Fg) \colon FD \to FE\) a projection pair in \(\mathcal{L}\)?
From \(g \circ f = 1_D\) we derive that \(Fg \circ Ff = F(g \circ f) = F1_D = 1_{FD}\).
However, from \(f \circ g \sqsubseteq 1_E\) it cannot be deduced that \(Ff \circ Fg \sqsubseteq 1_{FE}\), unless \(F\) is monotonic, when seen as a map between hom--sets. This leads to the following definition.

\begin{dfn}
  A functor \(F \colon \mathcal{C} \to \mathcal{L}\) between \(\mathbf{O}\)--categories is \emph{locally monotonic} iff, for all \(A, B \in \obj{\mathcal{C}}\), the map \(\mathcal{C}(A,B) \overset{F}{\to}\mathcal{L}(FA,FB)\) is monotonic.
\end{dfn}

Actually, local monotonicity ensures that, given a functor between \(\mathbf{O}\)--categories, even its restriction to the subcategories with embeddings as morphisms is a functor.

\paragraph{Continuity of \(F^\textup{E}\)}
The initiality theorem may help us in establishing an easy criterion to determine the \(\omega{}\)--continuity of \(F^\textup{E}\).
Assume that all the conditions of Theorem~\ref{thm:3-2} are satisfied, and that \(F \colon \mathcal{C} \to \mathcal{L}\) is a locally monotonic function.
Then, \(F^\textup{E} \colon \mathcal{C}^\textup{E} \to \mathcal{L}^\textup{E}\) is also a functor and \((F^\textup{E}D, \lbrace F^\textup{E}\alpha_i\rbrace_{i \in \mathbb{N}})\) is a cocone for the \(\omega{}\)--chain \(\Delta^\prime = \lbrace F^\textup{E} D_i, F^\textup{E}f_i \rbrace_{i \in \mathbb{N}}\) in \(\mathcal{L}^\textup{E}\).
Since \((D,\alpha)\) is a \(\Delta{}\)--colimit, the initiality theorem states that \(\bigsqcup_{n \in \mathbb{N}}(\alpha_n \circ \alpha_n^\star) = 1_D\).
Thus,
\begin{equation*}
  \begin{split}
    1_FD &= F1_D = F\left(\bigsqcup_{n \in \mathbb{N}}(\alpha_n \circ \alpha_n^\star)\right) \overset{\dagger}{=} \bigsqcup_{n \in \mathbb{N}}\left(F(\alpha_n \circ \alpha_n^\star)\right) = \bigsqcup_{n \in \mathbb{N}}(F\alpha_n \circ F\alpha_n^\star) \\
    &= \bigsqcup_{n \in \mathbb{N}}(F^\textup{E}\alpha_n \circ F^\textup{E}\alpha_n^\star)
  \end{split}
\end{equation*}
shows that \((F^\textup{E}D, \lbrace F^\textup{E}\alpha_i\rbrace_{i \in \mathbb{N}})\) is a \(\Delta^\prime{}\)--colimit; it is important to observe that the equality (\(\dagger{}\)) holds only when \(F\) is \(\omega{}\)--continuous, when viewed as a morphism map.

\begin{dfn}
  A functor \(F \colon \mathcal{C} \to \mathcal{L}\) between \(\mathbf{O}\)--categories is \emph{locally continuous} iff, for all \(A, B \in \obj{\mathcal{C}}\), the map \(\mathcal{C}(A,B) \overset{F}{\to}\mathcal{L}(FA,FB)\) is \(\omega{}\)--continuous.
\end{dfn}

Examples of locally continuous functors are the identity and the projection functors.

An immediate and importance consequence of this reasoning is contained in the following result.

\begin{thm}[Continuity theorem]
  Let \(\mathcal{C}\), \(\mathcal{L}\) be \(\mathbf{O}\)--categories and \(F \colon \mathcal{C} \to \mathcal{L}\) a locally continuous functor.
  Suppose that \(\mathcal{C}\) is localized.
  Then, \(F^\textup{E} \colon \mathcal{C}^\textup{E} \to \mathcal{L}^\textup{E}\) is \(\omega{}\)--cocontinuous.
\end{thm}

\subsection{Solution of a fixed point equation}

Let \(F\) be a locally continuous endofunctor over a localized \(\mathbf{O}\)--category \(\mathcal{C}\).
In addition, suppose that \(\mathcal{C}^\textup{E}\) is an \(\omega{}\)--category, with initial object \(\bot{}\).
Then, the following hold:
\begin{enumerate}
  \item \(F^\textup{E}\) is an \(\omega{}\)--cocontinuous endofunctor over \(\mathcal{C}^\textup{E}\);
  \item \(F^\textup{E}\) has an initial fixed point \((A,\alpha)\), where \(\alpha \colon FA \to A\) is iso in \(\mathcal{C}^\textup{E}\);
  \item \((A, \lbrace \mu_n \colon F^n\bot \to A \rbrace_{n \in \mathbb{N}^\star})\) is a colimit for a certain \(\omega{}\)--chain \(\Delta{}\) in \(\mathcal{C}^\textup{E}\);\footnote{Remember that \(F^\textup{E}\bot = F\bot{}\).}
  \item By functoriality of \(F^\textup{E}\), also \((FA, \lbrace F^\textup{E}\mu_{n-1} \colon F^n\bot \to A \rbrace_{n \in \mathbb{N}^\star})\) is a \(\Delta{}\)--colimit.
\end{enumerate}
Since \((A,\mu)\) is a \(\Delta{}\)--colimit, the initiality theorem implies
\begin{equation*}
  \bigsqcup_{n \in \mathbb{N}^\star}(\mu_n \circ \mu_n^\star) = 1_A.
\end{equation*}
Moreover, the mediating morphism from \(A\) to \(FA\) is given by
\begin{equation*}
  f := \bigsqcup_{n \in \mathbb{N}^\star}(F^\textup{E}\mu_{n-1} \circ \mu_n^\star).
  % \left(\bigsqcup_{n \in \mathbb{N}^\star}(F^\textup{E}\mu_{n-1} \circ \mu_n^\star),\bigsqcup_{n \in \mathbb{N}^\star}(\mu_n \circ F(\mu_{n-1}^\star))\right).% = \left(\bigsqcup_{n \in \mathbb{N}^\star}(F^\textup{E}\mu_{n-1} \circ \mu_n^\star), \bigsqcup_{n \in \mathbb{N}^\star}(\mu_n \circ F(\mu_{n-1}^\star))\right)
\end{equation*}

\begin{center}
  \begin{tikzcd}
    \bot \arrow[r, "!"] &%chktex 18
    F\bot \arrow[r, "F!"] &%chktex 18
    \dots \arrow[r] &
    F^n\bot \arrow[rr, "F^n!"]%chktex 18
        \arrow[dr, "\mu_n" description]%chktex 18
        \arrow[ddr, "F^\textup{E}\mu_n" description] & &%chktex 18
    F^{n+1} \arrow[r]%
            \arrow[dl, "\mu_{n+1}" description]%chktex 18
            \arrow[ddl, "F^\textup{E}\mu_{n+1}" description] & \dots{}\\%chktex 18
    & & & & A \arrow[d, "{f}" description]& \\%chktex 18
    & & & & FA &
  \end{tikzcd}
\end{center}
