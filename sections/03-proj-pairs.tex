\section{\(\mathbf{O}\)--categories and projection pairs}

In a general setting, verifying whether a given category is \(\omega\)--cocomplete --- and, in a similar fashion, checking if a functor is \(\omega\)--cocontinuous --- is not an easy task. %chktex 21
Nevertheless, there is a class of categories, first introduced by Wand in~\cite{Wand1979}, which structure allows for easily verifiable criteria to arise.
These so--called \(\mathbf{O}\)--categories include almost every category that occurs in semantics.%chktex 8
The results shown in this section are taken from~\cite{Smyth1982}.

\begin{dfn}
  An \emph{\(\mathbf{O}\)--category} is a category which satisfies the following conditions:
  \begin{enumerate}
    \item every hom--set is a partial order~\footnote{The partial order on a hom--set in an \(\mathbf{O}\)--category will hereafter be denoted by \(\sqsubseteq{}\).} in which every ascending %chktex 8
    \(\omega\)--chain has a least upper bound; %chktex 21
    \item composition of morphisms is an \(\omega\)--continuous operation with respect to that partial order. %chktex 21
  \end{enumerate}
\end{dfn}

In an \(\mathbf{O}\)--category, it is possible to consider \emph{projection pairs}. These turn out to have some useful applications, which we are going to show in later sections.

\begin{dfn}
  Let \(\mathcal{C}\) be an \(\mathbf{O}\)--category. Then, every tuple \((f,f^\star)\) satisfying
  \begin{align}
    f &\colon A \to B & f^\star \circ f = 1_A \quad  \\
    f^\star &\colon B \to A & f \circ f^\star \sqsubseteq 1_B \quad
  \end{align}
  with \(A,B \in \obj{\mathcal{C}}\) is called a \emph{projection pair}, where \(f\) is an \emph{embedding} and \(f^\star\) a \emph{projection}.%chktex 21
\end{dfn}

An interesting feature of projection pairs is that they're uniquely determined by one of their components: given an embedding, we can thereby take \emph{the} associated projection.
The following lemma justifies what has just been asserted.

\begin{lem}
  Let \((f,f^\star)\) and \((g,g^\star)\) be projection pairs from \(A\) to \(B\).
  Then,
  \begin{equation*}
    f \sqsubseteq g \iff g^\star \sqsubseteq f^\star \qquad \text{and} \qquad
    f = g \iff f^\star = g^\star.
  \end{equation*}
\end{lem}
\begin{proof}
  Assume that \(f \sqsubseteq h\). Since composition is \(\omega{}\)--continuous --- hence monotonic --- we find
  \begin{equation*}
    g^\star = 1_A \circ g^\star = (f^\star \circ f) \circ g^\star
    \sqsubseteq (f^\star \circ g) \circ g^\star \sqsubseteq f^\star \circ 1_B = f^\star.
  \end{equation*}
  Similarly, we prove the opposite implication. Finally,
  \begin{equation*}
    f = g \iff f \sqsubseteq g \land g \sqsubseteq f
    \iff g^\star \sqsubseteq f^\star \land f^\star \sqsubseteq g^\star \iff f^\star = g^\star.\qedhere
  \end{equation*}
\end{proof}

For each embedding \(f\), we will denote by \(f^\star\) its associated projection. %chktex 21
In addition, for an \(\mathbf{O}\)--category \(\mathcal{C}\), we denote by \(\mathcal{C}^\textup{E}\) its subcategory consisting of all its objects, but with as morphisms only the embeddings.
In fact, the composition of embeddings is still an embedding: given embeddings \(f\), \(g\) the arrow \(g \circ f\) is an embedding with corresponding projection \({(g \circ f)}^\star = f^\star \circ g^\star\). %chktex 21
In a similar fashion, it is possible to define \(\mathcal{C}^\textup{P}\), the subcategory of \(\mathcal{C}\) which morphisms are projections.

\subsection{The initiality theorem}

A first, important result, concerning colimits in \(\mathbf{O}\)--categories, is stated in the initiality theorem, that provides a simple way to check whether a cocone is a colimit in a subclass of \(\mathbf{O}\)--categories.

\begin{lem}\label{lem:3-1}
  Let \(\mathcal{C}\) be an \(\mathbf{O}\)--category, \(\Delta = {\langle (D_n, f_n) \rangle}_{n \in \mathbb{N}}\) an \(\omega\)--chain in \(\mathcal{C}^\textup{E}\) and \((D,\alpha)\), \((E,\beta)\) two \(\Delta\)--cocones. %chktex 21
  Then, the following facts hold:
  \begin{enumerate}
    \item \({\langle \alpha_n \circ \beta_n^\star \rangle}_{n \in \mathbb{N}}\) is an ascending \(\omega\)--chain in \(\mathcal{C}(E,D)\); %chktex 21
    \item if \(f = \bigsqcup_{n \in \mathbb{N}}(\alpha_n \circ \beta_n^\star)\), then \(f\) is a morphisms of cocones from \((E,\beta)\) to \((D,\alpha)\):
    \begin{equation*}
      \forall n \in \mathbb{N}.\: f \circ \beta_n = \alpha_n.
    \end{equation*}
  \end{enumerate}
\end{lem}
\begin{proof}
  Let \(n \in \mathbb{N}\) such that \(n \ge 0\). Then,
  \begin{equation*}
    \begin{split}
      \alpha_n \circ \beta_n^\star &=
      (\alpha_{n+1} \circ f_n) \circ {(\beta_{n+1} \circ f_n)}^\star \\
      &= \alpha_{n+1} \circ (f_n \circ f_n^\star) \circ \beta_{n+1}^\star
      \sqsubseteq \alpha_{n+1} \circ \beta_{n+1}.
    \end{split}
  \end{equation*}

  Passing on to the second fact, we should first note that \(f\) is well--defined, since \(\mathcal{C}\) is an \(\mathbf{O}\)--category. %chktex 8
  Furthermore, the deletion of an initial segment of an \(\omega\)--chain leaves its least upper bound unaffected. %chktex 21
  So,
  \begin{equation*}
    \begin{split}
      f \circ \beta_n &=
      \bigsqcup_{i \in \mathbb{N}}(\alpha_i \circ \beta_i^\star) \circ \beta_n
      = \bigsqcup_{i \ge n}(\alpha_i \circ \beta_i^\star) \circ \beta_n \\
      &= \bigsqcup_{i \ge n}(\alpha_i \circ \beta_i^\star \circ \beta_n)
      = \bigsqcup_{i \ge n}(\alpha_i \circ \beta_i^\star \circ \beta_i \circ f_i \circ f_{i-1} \circ \dotsc \circ f_n) \\
      &= \bigsqcup_{i \ge n}(\alpha_i \circ f_i \circ f_{i-1} \circ \dotsc \circ f_n)
      = \bigsqcup_{i \ge n}(\alpha_n) = \alpha_n.
    \end{split}
  \end{equation*}\qedhere
\end{proof}

A corner--case of the Lemma~\ref{lem:3-1} occurs when the two cocones coincide, i.e. \((E,\beta) = (D,\alpha)\). %chktex 8
Of course, \(f = \bigsqcup_{n \in \mathbb{N}}(\alpha_n \circ \alpha_n^\star) \sqsubseteq 1_D\).
Is it possible to make additional claims, if equality is reached?

\begin{thm}\label{thm:3-1}
  Let \(\mathcal{C}\) be an \(\mathbf{O}\)--category, \(\Delta = {\langle (D_n,f_n) \rangle}_{n \in \mathbb{N}}\) an \(\omega\)--chain in %chktex 21
  \(\mathcal{C}^\textup{E}\) and \((D,\alpha)\) a \(\Delta\)--cocone. %chktex 21
  Suppose that \(\bigsqcup_{n \in \mathbb{N}}(\alpha_n \circ \alpha_n^\star) = 1_D\).
  Then, \((D,\alpha)\) is a \(\Delta\)--colimit. %chktex 21
  Moreover, if \((E,\beta)\) is a \(\Delta\)--cocone, then %chktex 21
  \begin{equation*}
    f = \bigsqcup_{n \in \mathbb{N}}(\beta_n \circ \alpha_n^\star)
  \end{equation*}
  is the mediating morphism from \(D\) to \(E\) --- and its associated projection is \(f^\star = \bigsqcup_{n \in \mathbb{N}}(\alpha_n \circ \beta_n^\star)\).
\end{thm}
\begin{proof}
  First, we are going to show that \(f\) is an embedding from \(D\) to \(E\).
  \begin{align*}
    f^\star \circ f &\overset{\bullet}{=}
    \bigsqcup_{n \in \mathbb{N}}(\alpha_n \circ \beta_n^\star \circ \beta_n \circ \alpha_n^\star) =
    \bigsqcup_{n \in \mathbb{N}}(\alpha_n \circ \alpha_n^\star) \overset{\dagger}{=} 1_D; \\
    f \circ f^\star &\overset{\bullet}{=}
    \bigsqcup_{n \in \mathbb{N}}(\beta_n \circ \alpha_n^\star \circ \alpha_n \circ \beta_n^\star) =
    \bigsqcup_{n \in \mathbb{N}}(\beta_n \circ \beta_n^\star)
    \sqsubseteq 1_E.
  \end{align*}
  The equalities marked with \(\bullet\) are justified from the \(\omega\)--continuity of the composition, while \(\dagger\) follows from the hypothesis. %chktex 21

  As already seen in Lemma~\ref{lem:3-1}, \(f\) is a morphism of cocones.
  Now, what is left to prove is that \(f\) is the unique morphism from \((D,\alpha)\) to \((E,\beta)\), i.e. the unique morphism \(D\to E\)  such that, for all \(n \in \mathbb{N}\), \(f \circ \alpha_n = \beta_n\).
  Let \(\eta\) be another such morphism. Then, %chktex 21
  \begin{equation*}
    \begin{split}
      \eta &= \eta \circ 1_D
      = \eta \circ \bigsqcup_{n \in \mathbb{N}}(\alpha_n \circ \alpha_n^\star) = \bigsqcup_{n \in \mathbb{N}}(\eta \circ \alpha_n \circ \alpha_n^\star) \\
      &= \bigsqcup_{n \in \mathbb{N}}(\beta_n \circ \alpha_n^\star)
      = f.
    \end{split}
  \end{equation*} \qedhere
\end{proof}

It would be desirable to have a converse result, with respect to Theorem~\ref{thm:3-1}, namely, showing that \((D,\alpha)\) is a colimit if and only if the identity morphism \(1_D\) equals \(\bigsqcup_{n \in \mathbb{N}}(\alpha_n \circ \alpha_n^\star)\).

\begin{dfn}\label{dfn:localized}
  An \(\mathbf{O}\)--category \(\mathcal{C}\) is called \emph{localized} if and only if for any \(\omega\)--chain \(\Delta\) in \(\mathcal{C}^\textup{E}\) and any \(\Delta\)--colimit \((D,\alpha)\) there exist an object \(E \in \obj{\mathcal{C}}\) and a projection pair \((g,g^\star) \colon E \to D\) such that %chktex 21
  \begin{equation*}
    \bigsqcup_{n \in \mathbb{N}}(\alpha_n \circ \alpha_n^\star) = g \circ g^\star.
  \end{equation*}
\end{dfn}

Being \((g,g^\star)\) a projection pair, \(g \circ g^\star \sqsubseteq 1_D\) vacuously holds. What if \(g \circ g^\star = 1_D\)?

\begin{thm}[Initiality theorem]\label{thm:3-2}
  Let \(\mathcal{C}\) be a localized \(\mathbf{O}\)--category.
  Suppose that \(\Delta := {\langle(D_n,f_n)\rangle}_{n \in \mathbb{N}}\) is an \(\omega\)--chain in \(\mathcal{C}^\textup{E}\) and \((D,\alpha)\) a \(\Delta\)--cocone. %chktex 21
  The following are equivalent:
  \begin{enumerate}
    \item \((D,\alpha)\) is a \(\Delta{}\)--colimit;
    \item \(\bigsqcup_{n \in \mathbb{N}}(\alpha_n \circ \alpha_n^\star) = 1_D\).
  \end{enumerate}
\end{thm}
\begin{proof}[Proof Outline]
  The implication \(2 \to 1\) had already been proved in Theorem~\ref{thm:3-1}.
  Let \(E \in \obj{\mathcal{C}}\) and \(g \in \mathcal{C}^\textup{E}(E,D)\) such that \(\bigsqcup_{n \in \mathbb{N}}(\alpha_n \circ \alpha_n^\star) = g \circ g^\star\). %chktex 21
  To prove that \(g \circ g^\star = 1_D\), we should proceed in the following way:
  \begin{enumerate}
    \item By defining, for all \(n \in \mathbb{N}\), the morphism \(\beta_n := g \circ \alpha_n\), we should verify that \(\beta_n\) is indeed an embedding from \(D_n\) to \(E\) and that \((E,\beta)\) is a \(\Delta\)--cocone. %chktex 21
    \item After that, we should notice that the mediating morphism \(\eta \colon D \to E\) is the inverse of \(g\), i.e. \(g \circ \eta = 1_D\).
    This allows us to conclude that \(g \circ g^\star = 1_D\).
    Indeed, if the limit--case \(g \circ g^\star = 1_D\) is reached, then \(E\) and \(D\) are isomorphic and \(g^\star = g^{-1}\). %chktex 8
    Conversely, if \((g,g^\star)\) has an inverse \(\eta\), i.e. \((g,g^\star) \circ (\eta,\eta^\star) = (1_D,1_D)\), then %chktex 21
    \begin{equation*}
        1_D = 1_D \circ 1_D = g \circ \eta \circ \eta^\star \circ g^\star
        \sqsubseteq g \circ g^\star;
    \end{equation*}
    this, jointly with \(g \circ g^\star \sqsubseteq 1_D\), leads to \(g \circ g^\star = 1_D\). \qedhere
  \end{enumerate}
\end{proof}

To test whether an \(\mathbf{O}\)--category is localized, it would be desirable to have a simpler criterion. This can be obtained, thanks to idempotents and splits. Let \(\mathcal{C}\) be a category and \(f \colon D \to D\) an arrow, with \(D \in \obj{\mathcal{C}}\). Then, \(f\) is called an \emph{idempotent} when \(f \circ f = f\); it is called \emph{split} when there exists another object \(E\) and morphisms \(g \colon D \to E\), \(h \colon E \to D\) such that \(f = h \circ g\) and \(g \circ h = 1_E\).
\begin{center}
  \begin{tikzcd}
    & E \arrow[r, "1_E"] \arrow[d, "h"] & E \\%chktex 18
    D \arrow[r, "f"] \arrow[ur, "g"] & D \arrow[ur, "g"] &%chktex 18
  \end{tikzcd}
\end{center}

Note that any split is idempotent.\footnote{Indeed, \(f \circ f = h \circ g \circ h \circ g = h \circ 1_E \circ g = f\).}

\begin{prp}
  If every idempotent in an \(\mathbf{O}\)--category \(\mathcal{C}\) is split, then \(\mathcal{C}\) is localized.
\end{prp}
\begin{proof}
  Let \((D,\alpha)\) be a cocone for an \(\omega{}\)--chain \(\Delta\) in \(\mathcal{C}^\textup{E}\). %chktex 21
  Then, the morphism \(f := \bigsqcup_{n \in \mathbb{N}}(\alpha_n \circ \alpha_n^\star)\) is idempotent:
  \begin{equation*}
    \begin{split}
      f \circ f &= (\bigsqcup_{n \in \mathbb{N}}(\alpha_n \circ \alpha_n^\star)) \circ (\bigsqcup_{n \in \mathbb{N}}(\alpha_n \circ \alpha_n^\star)) \\
      &= \bigsqcup_{n \in \mathbb{N}}(\alpha_n \circ \alpha_n^\star \circ \alpha_n \circ \alpha_n^\star)
      = \bigsqcup_{n \in \mathbb{N}}(\alpha_n \circ \alpha_n^\star) = f.
    \end{split}
  \end{equation*}
  Thanks to the assumption that every idempotent is split in \(\mathcal{C}\), there exist an object \(E\) and two morphisms \(g \colon D \to E\), \(h \colon E \to D\) such that \(g \circ h = 1_E\) and \(h \circ g = f \sqsubseteq 1_D\).
  Thus, \((g,h)\) is a projection pair from \(E\) to \(D\) with \(f = h \circ g\).
  This makes us conclude that \(\mathcal{C}\) is localized.
\end{proof}

\subsection{The continuity theorem}

Let \(\mathcal{C}\) and \(\mathcal{L}\) be \(\mathbf{O}\)--categories. Suppose that \(F \colon \mathcal{C} \to \mathcal{L}\) is a functor.
Which conditions should we impose, to ensure that a functor \(F^\textup{E} \colon \mathcal{C}^\textup{E} \to \mathcal{L}^\textup{E}\) can be induced?

One would start by define \(F^\textup{E}\) as follows:
\begin{align*}
  F^\textup{E}(A) &= F(A), A \in \obj{\mathcal{C}^\textup{E}} \\
  F^\textup{E}(f) &= F(f), f\:\text{a morphism of}\:\mathcal{C}^\textup{E}.
\end{align*}
Is \(F(f)\) a morphism in \(\mathcal{L}^\textup{E}\)?
That is, assuming \((f,g) \colon D \to E\) is a projection pair in \(\mathcal{C}\), is \((F(f),F(g)) \colon FD \to FE\) a projection pair in \(\mathcal{L}\)?
From \(g \circ f = 1_D\) we derive that \(F(g) \circ F(f) = F(g \circ f) = F(1_D) = 1_{F(D)}\).
However, from \(f \circ g \sqsubseteq 1_E\) it cannot be deduced that \(F(f) \circ F(g) \sqsubseteq 1_{F(E)}\), unless \(F\) is monotonic, when seen as a map between hom--sets. %chktex 8
This leads to the following definition.

\begin{dfn}
  A functor \(F \colon \mathcal{C} \to \mathcal{L}\) between \(\mathbf{O}\)--categories is \emph{locally monotonic} iff, for all \(A, B \in \obj{\mathcal{C}}\), the map \(\mathcal{C}(A,B) \overset{F}{\to}\mathcal{L}(FA,FB)\) is monotonic.
\end{dfn}

Actually, local monotonicity ensures that, given a functor between \(\mathbf{O}\)--categories, even its restriction to the subcategories with embeddings as morphisms is a functor.

\paragraph{Continuity of \(F^\textup{E}\)}
The initiality theorem may help us in establishing an easy criterion to determine the \(\omega\)--continuity of \(F^\textup{E}\). %chktex 21
Assume that all the conditions of Theorem~\ref{thm:3-2} are satisfied, and that \(F \colon \mathcal{C} \to \mathcal{L}\) is a locally monotonic function.
Then, \(F^\textup{E} \colon \mathcal{C}^\textup{E} \to \mathcal{L}^\textup{E}\) is also a functor and \((F^\textup{E}D, {\langle F^\textup{E}\alpha_n\rangle}_{n \in \mathbb{N}})\) is a cocone for the \(\omega\)--chain %chktex 21
\(\Delta^\prime = {\langle F^\textup{E}(D_n), F^\textup{E}(f_n)\rangle}_{n \in \mathbb{N}}\) in \(\mathcal{L}^\textup{E}\).
Since \((D,\alpha)\) is a \(\Delta\)--colimit, the initiality theorem states that \(\bigsqcup_{n \in \mathbb{N}}(\alpha_n \circ \alpha_n^\star) = 1_D\). %chktex 21
Thus,
\begin{equation*}
  \begin{split}
    1_{F(D)} &= F(1_D) = F\left(\bigsqcup_{n \in \mathbb{N}}(\alpha_n \circ \alpha_n^\star)\right) \overset{\dagger}{=} \bigsqcup_{n \in \mathbb{N}}\left(F(\alpha_n \circ \alpha_n^\star)\right) \\ &= \bigsqcup_{n \in \mathbb{N}}(F(\alpha_n) \circ F(\alpha_n^\star))
    = \bigsqcup_{n \in \mathbb{N}}(F^\textup{E}(\alpha_n) \circ {F^\textup{E}(\alpha_n)}^\star)
  \end{split}
\end{equation*}
shows that \((F^\textup{E}(D), {\langle F^\textup{E}(\alpha_n)\rangle}_{n \in \mathbb{N}})\) is a \(\Delta^\prime\)--colimit; %chktex 21
it is important to observe that the equality (\(\dagger{}\)) holds only when \(F\) is \(\omega\)--continuous, when viewed as a morphism map. %chktex 21

\begin{dfn}
  A functor \(F \colon \mathcal{C} \to \mathcal{L}\) between \(\mathbf{O}\)--categories is \emph{locally continuous} iff, for all \(A, B \in \obj{\mathcal{C}}\), the map \(\mathcal{C}(A,B) \overset{F}{\to}\mathcal{L}(FA,FB)\) is \(\omega{}\)--continuous.
\end{dfn}

Examples of locally continuous functors are the identity and the projection functors.

An immediate and importance consequence of this reasoning is contained in the following result.

\begin{thm}[Continuity theorem]\label{thm:cont}
  Let \(\mathcal{C}\), \(\mathcal{L}\) be \(\mathbf{O}\)--categories and \(F \colon \mathcal{C} \to \mathcal{L}\) a locally continuous functor.
  Suppose that \(\mathcal{C}\) is localized.
  Then, \(F^\textup{E} \colon \mathcal{C}^\textup{E} \to \mathcal{L}^\textup{E}\) is \(\omega\)--cocontinuous. %chktex 21
\end{thm}

\subsection{Solution of \(D \simeq F(D)\)}

As an application of the results seen so far, we are now ready to construct a canonical solution to a recursive domain equation.
For instance, suppose that we want to solve the equation \(D \simeq F(D)\), where \(D\) is a categorical object in \(\mathcal{C}\) and \(F\) is an endofunctor over \(\mathcal{C}\).

If we require \(F\) to be a locally continuous endofunctor and \(\mathcal{C}\) to be a localized \(\mathbf{O}\)--category, thanks to Theorem~\ref{thm:cont}, it is possible to induce an \(\omega\)--cocontinuous endofunctor \(F^\textup{E}\) over \(\mathcal{C}^\textup{E}\). %chktex 21
If we additionally assume that \(\mathcal{C}^\textup{E}\) is \(\omega\)--cocomplete, with an initial object \(\bot\), by the initial fixed point theorem, there exists an initial fixed point \(\fix := (A,\alpha)\), where \(\alpha \in \mathcal{C}^\textup{E}(F(A), A)\) is an isomorphism. %chktex 21

\paragraph{Expliciting the isomorphism.}
Recalling the details of Theorem~\ref{thm:init},
\begin{equation*}
  \Delta := {\langle(F^n(\bot),F^n(!))\rangle}_{n \in \mathbb{N}}
\end{equation*}
is an \(\omega\)--chain in \(\mathcal{C}^\textup{E}\) and there exists a \(\Delta\)--colimit \((A, {\langle \mu_n \colon F^n(\bot) \to A \rangle}_{n \ge 1})\); moreover, another \(\Delta\)--colimit is given by \((F^\textup{E}(A), {\langle F^\textup{E}(\mu_{n-1}) \colon F^n(\bot) \to A \rangle}_{n \ge 1})\). %chktex 21

The Initiality Theorem~\ref{thm:3-2} ensures that \(\bigsqcup_{n \ge 1}(\mu_n \circ \mu_n^\star) = 1_A\), while Theorem~\ref{thm:3-1} provides an isomorphism from \(A\) to \(F(A)\) as
\begin{equation*}
  f := \bigsqcup_{n \in \mathbb{N}}(F^\textup{E}(\mu_{n-1}) \circ \mu_n^\star).
\end{equation*}

% \begin{center}
%   \begin{tikzcd}
%     \bot \arrow[r, "!"] &%chktex 18
%     F\bot \arrow[r, "F!"] &%chktex 18
%     \dots \arrow[r] &
%     F^n\bot \arrow[rr, "F^n!"]%chktex 18
%         \arrow[dr, "\mu_n" description]%chktex 18
%         \arrow[ddr, "F^\textup{E}\mu_n" description] & &%chktex 18
%     F^{n+1} \arrow[r]%
%             \arrow[dl, "\mu_{n+1}" description]%chktex 18
%             \arrow[ddl, "F^\textup{E}\mu_{n+1}" description] & \dots{}\\%chktex 18
%     & & & & A \arrow[d, "{f}" description]& \\%chktex 18
%     & & & & FA &
%   \end{tikzcd}
% \end{center}
