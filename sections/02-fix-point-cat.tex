\section{Fixed point theory for categories}\label{sec-2}

Category theory provides a very general framework for studying
recursive equations of the form \(D \simeq F(D)\), where \(D\) ought to be a domain of some kind, while \(F\) prescribes a way of
combining objects of these kinds.
In this section, some notions of category theory are introduced.
Then, a solution method for recursive domain equations is presented, as a
systematic generalisation of the theory outlined in Section~\ref{sec-1}.

\subsection{From \(\omega\)--cpos to \(\omega\)--cocomplete categories}%chktex 21

\begin{dfn}
  A \emph{category} \(\mathcal{C}\) consists of:
  \begin{enumerate}
    \item a class \(\obj{\mathcal{C}}\), containing its \emph{objects};
    \item for each \(A, B \in \obj{\mathcal{C}}\),
    a class \(\mathcal{C}(A,B)\) of the \emph{morphisms} from \(A\) to \(B\), also called \emph{hom--set}; %chktex 8
    \item for each \(A,B,C \in \obj{\mathcal{C}}\), a map
    \(
      \circ \colon \mathcal{C}(B,C) \times \mathcal{C}(A,B) \to \mathcal{C}(A,C)
    \)
    called the \emph{composition};
    the composition is then required to be \emph{associative}, i.e. if \(f \colon A \to B\), \(g \colon B \to C\) and \(h \colon C \to D\) then \((h \circ g) \circ f = h \circ (g \circ f)\);
    \item for each object \(A\), the \emph{identity} morphism \(1_A \colon A \to A\) satisfying \(f \circ 1_A = f\) and \(1_A \circ g = g\) for \(f \colon A \to B\) and \(g \colon B \to A\).
    Diagrammatically, this amounts to the commutativity of
    \begin{center}
      \begin{tikzcd}
        A \arrow[loop left]{l}{1_A}
          \arrow[r, yshift=2pt, "f"] & B \arrow[l, yshift=-2pt, "g"]% chktex 18
      \end{tikzcd}
    \end{center}
  \end{enumerate}
\end{dfn}

Given a category \(\mathcal{C}\), its \emph{dual}
\(\mathcal{C}^\textup{op}\) is made by reverting the arrows, i.e.
\(\obj{\mathcal{C}} = \obj{\mathcal{C}^\textup{op}}\) and \(f \in%
\mathcal{C}(A,B)\) implies \(f \in \mathcal{C}^\textup{op}(B,A)\), for
all \(A,B \in \obj{\mathcal{C}}\);
if \(\circ{}\) is the composition map over \(\mathcal{C}\), then
\(\mathcal{C}^\textup{op}\) is provided with a composition map
\(\bullet\) defined as \(f \bullet g := g \circ f\), for all morphisms %chktex 21
\(f\), \(g\) in \(\mathcal{C}\) which can be composed in \(\mathcal{C}\). %chktex 21
A \emph{subcategory} \(\mathcal{L}\) of \(\mathcal{C}\) is a category formed by taking a subclass of \(\obj{\mathcal{C}}\) and a subset for each morphism set of \(\mathcal{C}\). % chktex 8
A subcategory is \emph{full} if and only if the hom--sets are preserved, %chktex 8
i.e. \(\mathcal{L}(A,B) = \mathcal{C}(A,B)\), \(A,B \in \obj{\mathcal{L}}\).

\paragraph{From posets to categories} Let \((C,\sqsubseteq)\) be a poset. Is it possible to think of a partial order in terms of categorical entities?
Clearly, each element of \(C\) can be seen as an object of a certain category \(\mathcal{C}\). Moving on, the order relation may be recast using morphisms: given \(x, y \in C\),
\begin{equation*}
  \mathcal{C}(x,y) =
  \begin{cases}
    \lbrace x \to y \rbrace & x \sqsubseteq y \\ \emptyset & \text{otherwise}
  \end{cases}
\end{equation*}
guarantees the existence of the identity morphism, thanks to reflexivity, and the compositionality of arrows, since the order relation is transitive.
However, there isn't a categorical counterpart of antisimmetry.
In fact, if two objects can be related each other by means of morphisms, that doesn't imply they are the same object.

Thus, the concept of \emph{isomorphism} is introduced: two object \(A\), \(B\) are isomorphic whenever there exist two morphisms \(f \colon A \to B\), \(g \colon B \to A\) such that \(f \circ g = 1_B\) and \(g \circ f = 1_A\). Diagrammatically, it corresponds to the commutativity of
\begin{center}
  \begin{tikzcd}
    A \arrow[loop left]{l}{1_A}
      \arrow[r, yshift=2pt, "f"]% chktex 18
    & B \arrow[l, yshift=-2pt, "g"] \arrow[loop right]{r}{1_B}% chktex 18
  \end{tikzcd}
\end{center}
The morphism \(f\) is called and \emph{isomorphism}, while \(f^{-1} := g\) is its \emph{inverse}.

The categorical analogue of the least element of a poset is an \emph{initial object}, that is, an object \(\bot \in \obj{\mathcal{C}}\) such that each morphism set from \(\bot\) to each object of \(\mathcal{C}\) contains exactly one arrow. %chktex 21
Note that the initial object of a category is unique up to isomorphism: given \(A\), \(B\) both initial objects in \(\mathcal{C}\), the diagram
\begin{center}
  \begin{tikzcd}
    A \arrow[loop left]{l}{1_A}
      \arrow[r, yshift=2pt, "f"]% chktex 18
      & B\arrow[l, yshift=-2pt, "g"]% chktex 18
      \arrow[loop right]{r}{1_B}
  \end{tikzcd}
\end{center}
commutes, because \(\mathcal{C}(A,A) = \lbrace 1_A \rbrace\), \(\mathcal{C}(B,B) = \lbrace 1_B \rbrace\), \(\mathcal{C}(A,B) = \lbrace f \rbrace\) and \(\mathcal{C}(B,A) = \lbrace g \rbrace\) are the unique morphisms involved here. %chktex 21

Moving on to ascending \(\omega\)--chains, one has to deal with the fact that there might be more than a morphism going from an object to another. %chktex 21
Therefore, an \emph{\(\omega\)--chain} in a category \(\mathcal{C}\) is defined as a sequence \({\langle (D_n,f_n) \rangle}_{n \in \mathbb{N}}\), where each \(D_n\) is an object of \(\mathcal{C}\) and \(f_n \in \mathcal{C}(D_n, D_{n+1})\), with \(n \in \mathbb{N}\), is the specified morphism. %chktex 21

Diagrammatically, an \(\omega\)--chain is represented as %chktex 21
\begin{center}
  \begin{tikzcd}
    D_0 \arrow[r, "f_0"] &%chktex 18
    D_1 \arrow[r, "f_1"] &%chktex 18
    \dotsc{} \arrow[r, "f_{n-1}"] &%chktex 18
    D_n \arrow[r, "f_n"] &%chktex 18
    D_{n+1} \arrow[r, "f_{n+1}"] &%chktex 18
    \dotsc{}
  \end{tikzcd}
\end{center}

As an upper bound ``dominates'' each element of an ascending \(\omega\)--chain, so the categorical counterpart should possess at least one morphism from each chain object to the upper bound object. %chktex 21
Then, commutativity of the generated diagram is additionally required, so that each path taken to reach the supremum object from each \(D_i\) is equivalent.
% because of the possible existence of more than one morphism from a chain object to the ``upper bound'' object.
A \emph{cocone} for an \(\omega\)--chain \(\Delta\) in \(\mathcal{C}\) is %chktex 21
then defined as a pair \((D, \delta)\), with \(D\) an object of
\(\mathcal{C}\) and \(\delta = {\langle\delta_n\rangle}_{n\in\mathbb{N}}\)
a sequence of morphisms satisfying the condition
\(\delta_n = \delta_{n+1} \circ f_n\) for each \(n \in \mathbb{N}\). Diagrammatically, this amounts to the commutativity of
  \begin{center}
    \begin{tikzcd}%[row sep=large, column sep=small, cramped]
      D_0 \arrow[r, "f_0"] \arrow[drr, "\delta_0" description]%chktex 18
      & D_1 \arrow[r, "f_1"] \arrow[dr, "\delta_1" description]%chktex 18
      & \dotsc{} \arrow[r]
      & D_n \arrow[r, "f_n"] \arrow[dl, "\delta_i" description]%chktex 18
      & D_{n+1}\arrow[dll,"\delta_{n+1}" description]\arrow[r,"f_{n+1}"]%chktex 18
       & \dotsc{} \\
      & & D & & &
    \end{tikzcd}
  \end{center}

Thanks to the notions introduced so far, it is now possible to define an equivalent concept of a least upper bound, in category theory.
This can be achieved by considering the category \(\mathcal{U}_\Delta\), which objects are the cocones of the \(\omega\)--chain \(\Delta\) and which morphisms sets are specified as %chktex 21
\begin{equation*}
  \mathcal{U}_\Delta((A,\alpha),(B,\beta)) := \lbrace f \colon A \to B \mid
  \forall n \in \mathbb{N}.\,\beta_n = \alpha_n \circ f \rbrace;
\end{equation*}
the initial object \((A, \alpha)\) of this category is called a \emph{colimit} of \(\Delta\), while the unique morphism from \((A,\alpha)\) to each \(\Delta\)--cocone \((B, \beta)\) is said \emph{mediating}. %chktex 21
A diagrammatical representation of a colimit is the following\footnote{Here, only a part of the diagram has been visualized; however, \(f\) acts on each morphism of \(\langle \alpha_n \rangle_n\) and \(\langle \beta_n \rangle_n\)}:
\begin{center}
  \begin{tikzcd}
    D_0 \arrow[r, "f_0"] &%chktex 18
    D_1 \arrow[r, "f_1"] &%chktex 18
    \dots \arrow[r] &
    D_n \arrow[rr, "f_n"]%chktex 18
        \arrow[dr, "\alpha_n" description]%chktex 18
        \arrow[ddr, "\beta_n" description] & &%chktex 18
    D_{n+1} \arrow[r]%
            \arrow[dl, "\alpha_{n+1}" description]%%chktex 18
            \arrow[ddl, "\beta_{n+1}" description] & \dots{}\\%chktex 18
    & & & \dots & A \arrow[d, dashed, "f"]& \dots{}\\%chktex 18
    & & & & B &
  \end{tikzcd}
\end{center}
Due to their definition as initial objects of \(\mathcal{U}_\Delta\), colimits are unique up to isomorphism. %chktex 21
Now, all the elements are given, in order to introduce the categorical structure replacing the \(\omega\)--cpos. %chktex 21
The definition given below is taken from~\cite{Hemerik1988}

\begin{dfn}
  An \emph{\(\omega\)--cocomplete} category is a category which has initial objects and in which each \(\omega\)--chain has a colimit. %chktex 21
\end{dfn}

\subsection{Functors and fixed points}

We can reasonably argue that a monotonic function over a \(\omega\)--cpo may be thought as a ``structure--preserving'' map: the order is preserved and \(\omega\)--chains are mapped into \(\omega\)--chain. %chktex 21 chktex 8
Moving to category theory, one may consider transformations that preserve some structural features, like commutativity of diagrams or identities.

\begin{dfn}
  A \emph{functor} from the category \(\mathcal{C}\) to the category \(\mathcal{L}\) consists of
  \begin{enumerate}
    \item a map \(\obj{\mathcal{C}} \to \obj{\mathcal{L}}\),
    \item for all \(A, B \in \obj{\mathcal{C}}\), a map \(\mathcal{C}(A,B) \to \mathcal{L}(F(A),F(B))\) that preserves identity and composition, i.e.
      if \(g \circ f\) is defined in \(\mathcal{C}\), then
            \(F(g \circ f) = F(g) \circ F(f)\), and
            \(F(1_A) = 1_{F(A)}\).
  \end{enumerate}
\end{dfn}

% \begin{rem}
%   Let \textsc{cat} be the category which objects are categories and morphisms are functors. This enables us to affirm that two categories are \emph{isomorphic} whenever they are related by a iso functor in \textsc{cat}.
% \end{rem}

Then, as continuous functions over \(\omega\)--cpos map suprema to suprema, so it should be required for a functor to map colimits to colimits. This justifies the following definition. %chktex 21

\begin{dfn}
  A functor \(F \colon \mathcal{C} \to \mathcal{L}\) is \emph{\(\omega\)--cocontinuous} iff it preserves colimits: that is, given an \(\omega\)--chain \({\langle (D_n,f_n) \rangle}_{n \in \mathbb{N}}\) in \(\mathcal{C}\) with colimit \((D,\delta)\), the tuple \((F(D),F(\delta))\) is a colimit for the \(\omega\)--chain \({\langle (F(D_n),F(f_n))\rangle}_{n \in \mathbb{N}}\) in \(\mathcal{L}\). %chktex 21
\end{dfn}

Examples of \(\omega\)--cocontinuous functors are the projection functors \(\pi_1 \colon \mathcal{C} \times \mathcal{L} \to \mathcal{C}\), \(\pi_2 \colon \mathcal{C} \times \mathcal{L} \to \mathcal{L}\) and the diagonal functor \(\Delta \colon \mathcal{C} \to \mathcal{C} \times \mathcal{C}\). %chktex 21
Also, each iso functor is \(\omega\)--cocontinuous. %chktex 21

To conclude the transfer from order theory to category theory, we provide the counterpart of prefixed and fixed points.

\begin{dfn}
  Let \(\mathcal{C}\) be a category and \(F\) be an endofunctor\footnote{An \emph{endofunctor} over a category is a functor from that category to itself.} on \(\mathcal{C}\).
  Then, an \emph{\(F\)--algebra} is a pair \((A,\alpha)\) where \(A \in \obj{\mathcal{C}}\) and \(\alpha \in \mathcal{C}(FA,A)\).
  A \emph{fixed point} of \(F\) is an \(F\)--algebra with \(\alpha\) an isomorphism.%chktex 21
\end{dfn}

\subsection{The initial fixed point theorem}

\(F\)--algebras, as should be expected from most of the mathematical objects introduced so far, can be thought of as objects of a category, whose morphisms are constructed in the following way: given two \(F\)--algebras \((A,\alpha)\) and \((B,\beta)\), a \emph{\(F\)--homomorphism} from \((A,\alpha)\) to \((B,\beta)\) is an arrow \(f\) such that
\begin{center}
  \begin{tikzcd}
    F(A) \arrow[r, "\alpha"] \arrow[d, "F(f)"]%chktex 18
    & A \arrow[d, "f"]\\%chktex 18
    F(B) \arrow[r, "\beta"] & B %chktex 18
  \end{tikzcd}
\end{center}
commutes.

Now, a generalisation of Lemma~\ref{lem:cpo-prefixed} is asserted and proved.
\begin{lem}
  Let \(F\) be an endofunctor on a category \(\mathcal{C}\).
  If \((A,\alpha)\) is the initial \(F\)--algebra, then \(\alpha{}\) is an isomorphism.
\end{lem}
\begin{proof}
  By applying \(F\) to \((A,\alpha)\), we first get another \(F\)--algebra \((F(A),F(\alpha))\), where \(F(\alpha) \colon F(F(A)) \to F(A)\); this implies, thanks to the initiality of \((A,\alpha)\), that there exists the unique \(F\)--homomorphism \(! \colon (A,\alpha) \to (F(A),F(\alpha))\).
  Then, we observe that the diagram
    \begin{center}
      \begin{tikzcd}
        F(A) \arrow[r, "F(!)"] \arrow[d, "\alpha"]%chktex 18
        & F(F(A)) \arrow[r, "F(\alpha)"] \arrow[d, "F(\alpha)"]%chktex 18
        & F(A) \arrow[d, "\alpha"]\\%chktex 18
        A \arrow[r, "!"] & F(A) \arrow[r, "\alpha"]& A%chktex 18
      \end{tikzcd}
    \end{center}
 commutes, because:
  \begin{description}
    \item[left cell] the arrow \(!{}\) is an \(F\)--homomorphism;
    \item[right cell] trivially commutes.
  \end{description}
  Next, we claim that \(!{}\) is the two--sided inverse of \(\alpha{}\), since %chktex 8
  \begin{align*}
    \alpha \circ ! &= 1_A \\
    ! \circ \alpha &= F(\alpha) \circ F(!) = F (\alpha \circ !) = F(1_A) = 1_{F(A)}.
  \end{align*}
  The first equality holds, because of the initiality of \((A,\alpha)\): this implies that there is a unique endomorphism over \((A,\alpha)\); the identity morphism is required to exist for each object in a category, thus the identity is the unique endomorphism over \((A,\alpha)\).
  Lastly, we can say that \(\alpha{}\) is an isomorphism.
\end{proof}

The main result of this section is the ``initial fixed point'' theorem, which generalizes what was stated in Theorem~\ref{thm:cpo-fixed}.

\begin{thm}[initial fixed point theorem]\label{thm:init}
  Let \(\mathcal{C}\) be an \(\omega\)--cocomplete category, \(F\) an \(\omega{}\)--cocontinuous endofunctor over \(\mathcal{C}\) and \(\bot{}\) an initial object of \(\mathcal{C}\). %chktex 21
  If \(!{}\) is the unique morphism from \(\bot{}\) to \(F(\bot{})\), then
  \begin{equation}
    \Delta := {\langle(F^n(\bot),F^n(!))\rangle}_{n \in \mathbb{N}}
  \end{equation}
  is an \(\omega{}\)--chain, and for each \(\Delta\)--colimit \((A,\mu)\) there exists a morphism \(\alpha \colon F(A) \to A\) such that \(\fix := (A,\alpha)\) is an initial fixed point of \(F\). %chktex 21
\end{thm}
\begin{proof}[Proof Outline]
  Here, a brief idea of the proof is given. For details, see~\cite{Hemerik1988}.
  To prove that \(\Delta{}\) is an \(\omega{}\)--chain, one should proceed by induction on \(n \in \mathbb{N}\), showing that \(F^n(!) \in \mathcal{C}(F^n(\bot),F^{n+1}(\bot))\).

  Next, to see that with each \(\Delta\)--colimit \((A,{\langle\mu_n \rangle}_{n\in\mathbb{N}})\) comes an isomorphism \(\alpha \colon F(A) \to A\) such that \((A,\alpha)\) is an \(F\)--algebra, it is needed to observe that both \((A,{\langle \mu_{n+1} \rangle}_{n\in\mathbb{N}})\) and \((F(A),{\langle\mu_{n+1} \rangle}_{n\in\mathbb{N}})\) are colimits of \(\Delta \setminus (\bot,!)\).
  Thanks to this, the mediating morphism \(\alpha{}\) between the two colimits is the desired isomorphism.

  Let \((B,\beta)\) be another fixed point of \(F\). The proof of the initiality of \(\fix\) in the category of fixed points of \(F\) requires some additional steps:
  \begin{enumerate}
    \item First, another cocone \((B,\nu)\) for \(\Delta{}\) is built, with
    \begin{equation*}
      \nu_n =
      \begin{cases}
        \text{the unique morphism}\: \bot \to B & n = 0 \\
        \beta \circ F(\nu_{n-1}) & n \ge 1;
      \end{cases}
    \end{equation*}
    \item The mediating morphism \(\xi \colon (A,\mu) \to (B, \nu)\) is then shown to be a morphism of fixed points from \(\fix\) to \((B,\beta)\), specifically, \(\xi = \beta \circ F(\xi) \circ \alpha^{-1}\);
    \item Finally, it is proved that \(\xi{}\) is the unique morphism from \(\fix\) to \((B,\beta)\).
  \end{enumerate}
\end{proof}
%
% \subsection{Summary}
%
% \begin{tabular}{cc}
%   \toprule
%   Posets & Categories \\
%   \midrule
%   Monotonic functions & Functors \\
%   Continuous functions & Colimit--preserving functors \\
%
%   Order relation & Morphisms \\
%   Least elements & Initial objects \\
%   Upper bounds & Co--cones \\
%   lubs & Co--limits \\
%   \bottomrule
% \end{tabular}
