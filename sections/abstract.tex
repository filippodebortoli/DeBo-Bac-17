\begin{abstract}
  Recursive domain equations hold a prominent role in denotational semantics, as well as in the field of domain theory.
  %Recursive domain equations play an important role in denotational semantics and type theory.
  In fact, the recursive specification of datatypes that are commonly encountered in computer science can be achieved by finding the least fixed point of certain functions, thus solving equations of the form \(D = F(D)\), where \(D\) is a domain.
  %In fact, they allow the definition of recursive program and, more important, recursive specification of datatypes.
  A suitable mathematical setting to describe and solve these equations is given by category theory.
  In this work, a canonical construction for solutions of least fixed point equations over categorical domains will be introduced, with a focus on a particular type of categories where morphisms do satisfy certain additional conditions.
  Namely, a brief overview of least fixed point theory for cpos and categories will be presented.
  After that, a dissertation over \(\mathbf{O}\)--categories and solution of recursive equations in such categories will be held.
  Finally, it will be shown how main features of programming languages'types fit into this framework.
\end{abstract}
