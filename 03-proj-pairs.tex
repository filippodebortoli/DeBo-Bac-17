\section{\(\mathbf{O}\)--categories}

This section goes along with the work presented in \cite{MR677666}.

\begin{dfn}
  A category is an \emph{\(\mathbf{O}\)--category} if and only if:
  \begin{enumerate}
    \item every hom--set is a partial order in which every ascending \(\omega\)--chain has a least upper bound, and
    \item composition of morphisms is an \(\omega\)--continuous operation with respect to this partial order.
  \end{enumerate}
\end{dfn}

\begin{dfn}
  Let \(\mathcal{C}\) be an \(\mathbf{O}\)--category. Then, every tuple \((f,g)\) satisfying
  \begin{align}
    f &\colon A \to B & g \circ f = \mathrm{id}_A \quad  \\
    g &\colon B \to A & f \circ g \sqsubseteq \mathrm{id}_B \quad 
  \end{align}
  with \(A,B \in \lvert\mathcal{C}\rvert\) is called a \emph{projection pair}, where \(f\) is an \emph{embedding} and \(g\) a \emph{projection}.
\end{dfn}

\begin{lem}
  Let \((f,g)\) and \((h,k)\) be projection pairs from \(A\) to \(B\) in the \(\mathbf{O}\)--category \(\mathcal{C}\). Then,
  \begin{equation*}
    f \sqsubseteq h \iff k \sqsubseteq g \qquad \text{and} \qquad
    f = h \iff g = k.
  \end{equation*}
\end{lem}
\begin{proof}
  Assume that \(f \sqsubseteq h\). Since composition is \(\omega\)--continuous --- hence monotonic --- we find
  \begin{equation*}
    k = \mathrm{id}_A \circ k = (g \circ f) \circ k
    \sqsubseteq (g \circ h) \circ k \sqsubseteq g \circ \mathrm{id}_B = g.
  \end{equation*}
  Similarly, we prove the opposite implication. Finally,
  \begin{equation*}
    f = h \iff f \sqsubseteq h \land h \sqsubseteq f
    \iff k \sqsubseteq g \land g \sqsubseteq k \iff g = k.
  \end{equation*}
\end{proof}

For any given \(\mathbf{O}\)--category \(\mathcal{C}\), we can define the category of its projection pairs \(\mathcal{C}_\textup{PR}\) in the following way:
\begin{enumerate}
  \item the \emph{objects} of \(\mathcal{C}_\textup{PR}\) are the same as \(\mathcal{C}\);
  \item each \emph{hom--set} \(\hom(A,B)\) corresponds to the set of all projection pairs from \(A\) to \(B\);
  \item the \emph{composition} of projection pairs \(\alpha \colon A \to B\) and \(\beta \colon B \to C\) is the pair \((\beta^\textup{L} \circ \alpha^\textup{L}, \alpha^\textup{R} \circ \beta^\textup{R})\) which is indeed a projection pair from \(A\) to \(C\), as shown in Diagram~\ref{03:dia-1};
  \item the \emph{identity} of \(\hom(A,A)\) is the pair \((\mathrm{id}_A,\mathrm{id}_A)\).
\end{enumerate}
\begin{figure}%[!ht]
  \begin{center}
    \begin{tikzcd}
      & C \arrow[d, "\beta^\textup{R}"] & \\
      A \arrow[r, "\alpha^\textup{L}"] &
      B \arrow[r, "\beta^\textup{L}"] \arrow[d, "\alpha^\textup{R}"] &
      C \arrow[ul, "\mathrm{id}_C" description]\\
      & A  \arrow[ul, "\mathrm{id}_A" description]&
    \end{tikzcd}
  \end{center}
  \caption{We can easily compute the compositions of the embedding \(\beta^\textup{L} \circ \alpha^\textup{L}\) and of the projection \(\alpha^\textup{R} \circ \beta^\textup{R}\) to verify that they form a projection pair from \(A\) to \(C\).} 
  \label{03:dia-1}
\end{figure}