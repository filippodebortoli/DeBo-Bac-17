\section{Fixed point theory for categories}

\begin{lem}
  Let \(\mathcal{C}\) be a category and \(F \colon \mathcal{C} \to \mathcal{C}\) an endofunctor.
  If \((A,\alpha)\) is an initial prefixed point of \(F\), then it is also an initial fixed point of \(F\).
\end{lem}
\begin{proof}
  Being \((A,\alpha)\) a prefixed point of \(F\), it follows that \((F(A),F(\alpha))\) --- where \(F(\alpha)\) goes from \(F(F(A))\) to \(F(A)\) --- is a prefixed point of \(F\).
  Moreover, thanks to the initiality of \((A,\alpha)\), we can consider the unique morphism \(f \colon (A,\alpha) \to (F(A),F(\alpha))\); both the cells of the diagram~\ref{02:diagram-1}
  \begin{figure}[!ht]
    \begin{center}
      \begin{tikzcd}
        F(A) \arrow[r, "F(f)"] \arrow[d, "\alpha"]
        & F(F(A)) \arrow[r, "F(\alpha)"] \arrow[d, "F(\alpha)"]
        & F(A) \arrow[d, "\alpha"]\\
        A \arrow[r, "f"] & F(A) \arrow[r, "\alpha"]& A
      \end{tikzcd}
    \end{center}
    \caption{The cell on the left commutes trivially; the right one, because \(\alpha\) is a morphism from \((F(A),F(\alpha))\) to \((A,\alpha)\). [\textbf{Da rivedere}]}
    \label{02:diagram-1}
  \end{figure}
  commute.
  Then, we can see that \(f\) is the inverse of \(\alpha\), since
  \begin{align}
    \alpha \circ f &= \mathrm{id}_A \\
    f \circ \alpha &= F(\alpha) \circ F(f) = F (\alpha \circ f) = F(\mathrm{id}_A) = \mathrm{id}_{F(A)}.
  \end{align}
  From this, we obtain that \((A,\alpha)\) is a fixed point of \(F\).
  Finally, we conclude that it is also initial: the category of fixed points of \(F\) is in fact a full subcategory of the category of prefixed points of \(F\).
\end{proof}

\paragraph{Da sistemare} qui ci starebbe bene un commento di introduzione al piatto forte della sezione\dots

\begin{thm}[initial fixed point theorem]
  Let \(\mathcal{C}\) be a complete category, \(F\) a continuous endofunctor of \(F\), and \(U\) an initial object of \(\mathcal{C}\). If \(u\) is the unique morphism from \(U\) to \(F(U)\), then
  \begin{equation}
    \Delta := \lbrace (F^n(U),F^n(u)) \rbrace_{n \in \mathbb{N}}
  \end{equation}
  is an \(\omega\)-chain, and for each colimit \((A,\mu)\) of \(\Delta\) there is a morphism \(\alpha \colon F(A) \to A\) such that \((A,\alpha)\) is an initial fixed point of \(F\).
\end{thm}
\begin{proof}
  Questa la scrivo la prossima volta.
\end{proof}