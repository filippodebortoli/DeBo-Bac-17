% arara: clean
% arara: pdflatex
% arara: biber
% arara: pdflatex
% arara: pdflatex
% arara: lmkclean
\documentclass[draft, titlepage]{article}

\usepackage[english]{babel}
\usepackage[T1]{fontenc}
\usepackage[utf8]{inputenc}

%\usepackage[cmintegrals,cmbraces]{newtxmath}
%\usepackage{ebgaramond}
%\usepackage{ebgaramond-maths}
%\usepackage[urw-garamond]{mathdesign}
%\usepackage{garamondx}
\usepackage[sc,osf]{mathpazo} % add possibly `sc` and `osf` options
\usepackage[euler-digits]{eulervm}

\usepackage[backend=biber]{biblatex}
  \addbibresource{bibliography.bib}
\usepackage{csquotes}


\usepackage{microtype}
%\usepackage{frontespizio}
\usepackage{tikz}
\usetikzlibrary{cd}
\usepackage{amsthm}
\usepackage{amsmath}
\usepackage{amssymb}
%\usepackage{mathpazo}
\usepackage[font={color=gray,small}, justification=centerlast, width=.75\textwidth]{caption}

\usepackage{notation}

\begin{document}

% \begin{frontespizio}
%   \Universita{Trento}
%   \Facolta{Scienze Matematiche, Fisiche e Naturali}
%   \Corso[Laurea]{Matematica}
%   \Titoletto{Tesi di laurea}
%   \Titolo{Recursive domain equations:\\
%   a categorical perspective}
%   \Candidato[166041]{Filippo De Bortoli}
%   \Relatore{Roberto Zunino}
%   \Annoaccademico{2015-2016}
% \end{frontespizio}

  \tableofcontents
  
  \section[Fixed point theory for cpos]%
{Fixed point theory for complete partial orders}

% Further explanations: Winskel, p.72.
Before getting our hands dirty with category theory, it would be appropriate to
introduce the well-established framework for the solution of recursive domain
equations based on complete partial orders.
These structures, along with continuous functions over them, allow us to properly treat the foundational aspects of denotational semantics.

\subsection[Complete partial orders]{Complete partial orders}

A \emph{partially ordered} set, or \emph{poset}, consists of a set endowed with a reflexive, antisymmetric and transitive binary relation.
Typical examples of posets are:
\begin{itemize}
  \item numerical sets, endowed with the standard order relation;
  \item given a set \(X\), the power set \(2^X\) with the inclusion relation.
\end{itemize}
In domain theory, we can visualize the elements of a partial order as points of information; thus, we can consider the \emph{information ordering}, denoted with the symbol ``\(\sqsubseteq\)'', which sorts out the points of a poset by the amount of information they convey.

Taking the investigation further, it would also be desirable to model steps of consequent computations, where each new object carries more information than the previous one.
In a partial order \((D, {}\sqsubseteq)\), this can be represented with an
\emph{ascending \(\omega\)--chain}, which is an increasing sequence
\begin{equation*}
  d_0 \sqsubseteq d_1 \sqsubseteq \dotsb \sqsubseteq d_n \sqsubseteq \dotsb{}
\end{equation*}
of elements of \(D\).
Following the analogy between semantical domains and mathematical objects, the least upper bound of such an \(\omega\)--chain, if it exists, would be that elements that combines anything but the information contained in the elements of the \(\omega\)--chain.

Those requirements, essential to succeed in the modelling of meaning domains, lead to the following definition.

\begin{dfn}
  A \emph{complete partial order}, o \emph{cpo}, is a partial order in which every ascending \(\omega\)--chain has a least upper bound.
  An \emph{\(\omega\)--cpo} is a complete partial order with a least element
  \(\bot\), called the \emph{bottom}.
\end{dfn}

\subsection[Continuous functions]{Continuous functions}

The first criterium to be rispected in the construction of a theoretical model for the description of programs written in a certain programming language is that
\begin{displayquote}
  The more input is provided, the more output is expected.
\end{displayquote}
How can this be obtained? In mathematical terms, the notion of \emph{monotonic} functions is introduced, namely functions of posets%
\(f \colon (D_1,{}\sqsubseteq_1) \to (D_2,{}\sqsubseteq_2)\) such that
\begin{equation*}
  \forall{}x,y \in D_1.\; x \sqsubseteq_1 y \implies f(x) \sqsubseteq_2 f(y).
\end{equation*}
Moreover, it is important that programs do behave in a (somehow) predictable way: going on with the computations, one would like to come to a conclusion.
This justifies the following definition.

\begin{dfn}
  A function of cpos \(f \colon D_1 \to D_2\) is \emph{continuous} whenever
  \begin{equation*}
    \bigsqcup_{n \in \mathbb{N}} f(d_n) = f\left(\bigsqcup_{n\in\mathbb{N}}d_n\right)
  \end{equation*}
  hold for each ascending \(\omega\)--chain \(\lbrace d_n \rbrace_{n \in \mathbb{N}} \subseteq D_1\).
\end{dfn}

\subsection[Fixed points]{Least fixed points and recursion}

Now that we've defined which structures underlie our mathematical framework, it is time to see how a formal foundation of denotational semantics is given, supporting both recursive programs and recursive data types, defined as least
fixed points of continuous functions over a complete partial order.
To back this claim up, we are going to state some fundamental results in the field.

\begin{lem}[least (pre)fixed point lemma]\label{lem:cpo-prefixed}
  Let \(f\) be a monotonic function from a poset \(D\) to itself.
  Then, if it exists, the least prefixed point of \(f\) is also its least fixed point.
\end{lem}
\begin{proof}
  First, if we assume that \(x\) is the least prefixed point of \(f\), then \(f(x)\) is also a prefixed point of \(f\), since \(f\) is monotonic:
  \begin{equation*}
    f(x) \sqsubseteq x \implies f(f(x)) \sqsubseteq f(x).
  \end{equation*}
  Moreover, we can also say that \(x \sqsubseteq f(x)\), being \(x\) the least element of the set of prefixed points of \(f\).
  Hence, \(x\) is a fixed point of \(f\).
  Last, we notice that each fixed point of \(f\) is also prefixed, so \(x\) is also the least fixed point of \(f\).\qedhere
\end{proof}

\begin{thm}[Fixed point theorem]\label{thm:cpo-fixed}
  Let \(f\) be a continuous function from an \(\omega\)--cpo \(D\) to itself.
  Then, \(\lbrace f^n(\bot) \rbrace_{n \in \mathbb{N}}\) is an ascending \(\omega\)--chain in \(D\) and
  \begin{equation}
    \fix(f) = \bigsqcup_{n \in \mathbb{N}} f^n(\bot)
  \end{equation}
  is the least fixed point of \(f\).  
\end{thm}

\begin{thm}[Knaster--Tarski]\label{thm:knaster-tarski}
  Suppose that \((D, {}\sqsubseteq)\) is a complete lattice, that is, a partial order which has greatest lower bounds of arbitrary subsets.
  Then, for a monotonic function from \(D\) to itself,
  \begin{equation}
    \fix(f) := \bigsqcap\,\,\lbrace d \in D \colon f(d) \sqsubseteq d \rbrace
  \end{equation}
  is the least fixed point of \(f\).
\end{thm}
  \section{Fixed point theory for categories}

\begin{lem}
  Let \(\mathcal{C}\) be a category and \(F \colon \mathcal{C} \to \mathcal{C}\) an endofunctor.
  If \((A,\alpha)\) is an initial prefixed point of \(F\), then it is also an initial fixed point of \(F\).
\end{lem}
\begin{proof}
  Being \((A,\alpha)\) a prefixed point of \(F\), it follows that \((F(A),F(\alpha))\) --- where \(F(\alpha)\) goes from \(F(F(A))\) to \(F(A)\) --- is a prefixed point of \(F\).
  Moreover, thanks to the initiality of \((A,\alpha)\), we can consider the unique morphism \(f \colon (A,\alpha) \to (F(A),F(\alpha))\); both the cells of the diagram~\ref{02:diagram-1}
  \begin{figure}[!ht]
    \begin{center}
      \begin{tikzcd}
        F(A) \arrow[r, "F(f)"] \arrow[d, "\alpha"]
        & F(F(A)) \arrow[r, "F(\alpha)"] \arrow[d, "F(\alpha)"]
        & F(A) \arrow[d, "\alpha"]\\
        A \arrow[r, "f"] & F(A) \arrow[r, "\alpha"]& A
      \end{tikzcd}
    \end{center}
    \caption{The cell on the left commutes trivially; the right one, because \(\alpha\) is a morphism from \((F(A),F(\alpha))\) to \((A,\alpha)\). [\textbf{Da rivedere}]}
    \label{02:diagram-1}
  \end{figure}
  commute.
  Then, we can see that \(f\) is the inverse of \(\alpha\), since
  \begin{align}
    \alpha \circ f &= \mathrm{id}_A \\
    f \circ \alpha &= F(\alpha) \circ F(f) = F (\alpha \circ f) = F(\mathrm{id}_A) = \mathrm{id}_{F(A)}.
  \end{align}
  From this, we obtain that \((A,\alpha)\) is a fixed point of \(F\).
  Finally, we conclude that it is also initial: the category of fixed points of \(F\) is in fact a full subcategory of the category of prefixed points of \(F\).
\end{proof}

\paragraph{Da sistemare} qui ci starebbe bene un commento di introduzione al piatto forte della sezione\dots

\begin{thm}[initial fixed point theorem]
  Let \(\mathcal{C}\) be a complete category, \(F\) a continuous endofunctor of \(F\), and \(U\) an initial object of \(\mathcal{C}\). If \(u\) is the unique morphism from \(U\) to \(F(U)\), then
  \begin{equation}
    \Delta := \lbrace (F^n(U),F^n(u)) \rbrace_{n \in \mathbb{N}}
  \end{equation}
  is an \(\omega\)-chain, and for each colimit \((A,\mu)\) of \(\Delta\) there is a morphism \(\alpha \colon F(A) \to A\) such that \((A,\alpha)\) is an initial fixed point of \(F\).
\end{thm}
\begin{proof}
  Questa la scrivo la prossima volta.
\end{proof}
  \section{\(\mathbf{O}\)--categories}

This section goes along with the work presented in \cite{smy-plot}.

\begin{dfn}
  A category is an \emph{\(\mathbf{O}\)--category} if and only if:
  \begin{enumerate}
    \item every hom--set is a partial order in which every ascending \(\omega\)--chain has a least upper bound, and
    \item composition of morphisms is an \(\omega\)--continuous operation with respect to this partial order.
  \end{enumerate}
\end{dfn}

\begin{dfn}
  Let \(\mathcal{C}\) be an \(\mathbf{O}\)--category. Then, every tuple \((f,g)\) satisfying
  \begin{align}
    f &\colon A \to B & g \circ f = \mathrm{id}_A \quad  \\
    g &\colon B \to A & f \circ g \sqsubseteq \mathrm{id}_B \quad 
  \end{align}
  with \(A,B \in \obj{\mathcal{C}}\) is called a \emph{projection pair}, where \(f\) is an \emph{embedding} and \(g\) a \emph{projection}.
\end{dfn}

\begin{lem}
  Let \((f,g)\) and \((h,k)\) be projection pairs from \(A\) to \(B\) in the \(\mathbf{O}\)--category \(\mathcal{C}\). Then,
  \begin{equation*}
    f \sqsubseteq h \iff k \sqsubseteq g \qquad \text{and} \qquad
    f = h \iff g = k.
  \end{equation*}
\end{lem}
\begin{proof}
  Assume that \(f \sqsubseteq h\). Since composition is \(\omega\)--continuous --- hence monotonic --- we find
  \begin{equation*}
    k = \mathrm{id}_A \circ k = (g \circ f) \circ k
    \sqsubseteq (g \circ h) \circ k \sqsubseteq g \circ \mathrm{id}_B = g.
  \end{equation*}
  Similarly, we prove the opposite implication. Finally,
  \begin{equation*}
    f = h \iff f \sqsubseteq h \land h \sqsubseteq f
    \iff k \sqsubseteq g \land g \sqsubseteq k \iff g = k.
  \end{equation*}
\end{proof}

For any given \(\mathbf{O}\)--category \(\mathcal{C}\), we can define the category of its projection pairs \(\mathcal{C}_\textup{PR}\) in the following way:
\begin{enumerate}
  \item the \emph{objects} of \(\mathcal{C}_\textup{PR}\) are the same as \(\mathcal{C}\);
  \item each \emph{hom--set} \(\hom(A,B)\) corresponds to the set of all projection pairs from \(A\) to \(B\);
  \item the \emph{composition} of projection pairs \(\alpha \colon A \to B\) and \(\beta \colon B \to C\) is the pair \((\beta^\textup{L} \circ \alpha^\textup{L}, \alpha^\textup{R} \circ \beta^\textup{R})\) which is indeed a projection pair from \(A\) to \(C\), as shown in Diagram~\ref{03:dia-1};
  \item the \emph{identity} of \(\hom(A,A)\) is the pair \((\mathrm{id}_A,\mathrm{id}_A)\).
\end{enumerate}
\begin{figure}%[!ht]
  \begin{center}
    \begin{tikzcd}
      & C \arrow[d, "\beta^\textup{R}"] & \\
      A \arrow[r, "\alpha^\textup{L}"] &
      B \arrow[r, "\beta^\textup{L}"] \arrow[d, "\alpha^\textup{R}"] &
      C \arrow[ul, "\mathrm{id}_C" description]\\
      & A  \arrow[ul, "\mathrm{id}_A" description]&
    \end{tikzcd}
  \end{center}
  \caption{We can easily compute the compositions of the embedding \(\beta^\textup{L} \circ \alpha^\textup{L}\) and of the projection \(\alpha^\textup{R} \circ \beta^\textup{R}\) to verify that they form a projection pair from \(A\) to \(C\).} 
  \label{03:dia-1}
\end{figure}
  \printbibliography
  
\end{document}
